%%%  TO FORMAT A PREPRINT To submit to a PCI.  If accepted then use this main.tex with the package of accepted article and search for [IF ACCEPTED] comments  %%%

\documentclass[a4paper]{article}


%%%   SET THE TITLE   %%%
\newcommand{\preprinttitle}{A Model-Based Framework for the Analysis of the Dated Archaeological Record}
%%%%%%%%%%%%%%%%%%%%%%%%%


%%%  SET THE LIST OF AUTHORS WITH CORRESPONDING AFFILIATIONS  use \& before last author %%%
\newcommand{\listauthors}{\raggedright 
Miguel de Navascués\textsuperscript{1,2} \&
Concetta Burgarella\textsuperscript{2}
}
%%%%%%%%%%%%%%%%%%%%


%%%  SET THE LIST OF AFFILIATIONS  %%%
\newcommand{\listinstitutions}{
\textsuperscript{1} CBGP, INRAE, CIRAD, IRD, Montpellier SupAgro, Univ Montpellier, Montpellier, France
\\
\textsuperscript{2} Human Evolution, Department of Organismal Biology, Uppsala University, Uppsala, Sweden
}
%%%%%%%%%%%%%%%%%%%%%%


%%%  SET THE 'CORRESPONDENCE TO'  %%%
\newcommand{\email}{miguel.navascues@inrae.fr}
%%%%%%%%%%%%%%%%%%%%%%

%%%  SET THE ABSTRACT %%%
\newcommand{\preprintabstract}{Write here the abstract.}
%%%%%%%%%%%%%%%%%%%%%%


%%%  SET THE KEYWORDS  %%%
\newcommand{\preprintkeywords}{likelihood-free inference; simulation; radiocarbon dating; demography; model test}
%%%%%%%%%%%%%%%%%%%%%%




%%% [IF ACCEPTED] UNCOMMENT TO CHOOSE A PCI %%%
%\newcommand{\PCI}{Peer Community In Zoology}
%\newcommand{\PCI}{Peer Community In Ecology}
%\newcommand{\PCI}{Peer Community In Evolutionary Biology}
%\newcommand{\PCI}{Peer Community In Paleontology}
%\newcommand{\PCI}{Peer Community In Archaeology}
%\newcommand{\PCI}{Peer Community In Animal Science}
%\newcommand{\PCI}{Peer Community In Genomics}
%\newcommand{\PCI}{Peer Community In Neuroscience}
%\newcommand{\PCI}{Peer Community In Mathematical and Computational Biology}
%\newcommand{\PCI}{Peer Community In Forest and Wood Sciences}
%\newcommand{\PCI}{Peer Community In Ecotoxicology and Environmental Chemistry}
%\newcommand{\PCI}{Peer Community In Infections}
%\newcommand{\PCI}{Peer Community In Registered Reports}
%\newcommand{\PCI}{Peer Community In Network Science}
%\newcommand{\PCI}{Peer Community In Microbiology}
%%%%%%%%%%%%%%%%%%%%%

%%%  [IF ACCEPTED] UNCOMMENT AND CHOOSE "yes" or "no" if your article contains data for which you provided the link to raw data, and {statistical script or computer code} %%%
%\newcommand{\opendata}{yes}
%\newcommand{\opencode}{yes}
%%%%%%%%%%%%%%%


%%%  [IF ACCEPTED] UNCOMMENT AND SET THE version of the preprint. Replace X by a number   %%%
%\newcommand{\version}{X}
%%%%%%%%%%%%%%%%%%


%%%  [IF ACCEPTED] UNCOMMENT AND SET THE DATE OF LATEST UPLOAD on the preprint server    %%%
%\newcommand{\datepub}{ddth Month yyyy}
%%%%%%%%%%%%%%


%%%  [IF ACCEPTED] UNCOMMENT AND SET THE RECOMMENDER(s) NAME(s)  %%%
%\newcommand{\recommender}{FirstName FamilyName}
%%%%%%%%%%%%%%%%%

%%%  [IF ACCEPTED] UNCOMMENT AND SET THE DOI of the RECOMMENDATION  %%%
%\newcommand{\DOIrecommendation}{xxx/xxx}
%%% for example 10.24072/pci.mcb.100003 %%%


%%%  [IF ACCEPTED] UNCOMMENT AND SET THE REVIEWERS' NAMES IF KNOWN and/or X anonymous reviewers  %%%
%\newcommand{\reviewers}{FirstName FamilyName and two anonymous reviewer}
%%%%%%%%%%%%%%%


%%%   [IF ACCEPTED] UNCOMMENT AND SET THE 'CITE AS' OF YOUR PREPRINT. paste the line we sent you by Email (XXX the "cite as") in place of the xxx  %%%
%\newcommand{\citeas}{xxx}
%%%%%%%%%%%%%%%%%

%%%%%%%%%%TO FORMAT A PREPRINT%%%%%%%%%%%%%%%%
%%%%%%%%%%%%%%%%%%%%%%%%%%%%%%%%%%%%%%%%%%%%%%
%%%%%%%%%%%%%%%%%%%%%%%%%%%%%%%%%%%%%%%%%%%%%%

\usepackage[margin=1in,marginparwidth=4.2cm,marginparsep=0.5cm]{geometry}

\usepackage{bm} % for bold symbols in math environment

\usepackage{marginnote}
\usepackage{pdflscape}
\usepackage{ifthen}
\reversemarginpar  % sets margin notes to the left
\usepackage{lipsum} % Required to insert dummy text
\usepackage{calc}
\usepackage{siunitx}
\usepackage[right]{lineno}
\usepackage{titlesec}
\usepackage{indentfirst}
%\usepackage[none]{hyphenat} % use only if there is a problem
% Use Unicode characters
\usepackage[utf8]{inputenc}
\usepackage[T1]{fontenc}
% Clean citations with biblatex
\usepackage[
backend=biber,
natbib=true,
sortcites=true,
defernumbers=true,
style=authoryear,
citestyle=authoryear-comp,
maxnames=99,
maxcitenames=2,
uniquename=init,
giveninits=true,
terseinits=true,
url=true
]{biblatex}
\DeclareNameAlias{default}{family-given}
\renewcommand*{\revsdnamepunct}{} % no comma between family and given names
\renewbibmacro{in:}{%
  \ifentrytype{article}{}{\printtext{\bibstring{in}\intitlepunct}}} % remove 'In:' before journal name
\DeclareFieldFormat[article]{pages}{#1} % remove pp.
\AtEveryBibitem{\ifentrytype{article}{\clearfield{number}}{}} % don't print issue numbers
\DeclareFieldFormat[article, inbook, incollection, inproceedings, misc, thesis, unpublished]{title}{#1} % title without quotes
\usepackage{csquotes}
\RequirePackage[english]{babel} % must be called after biblatex
\addbibresource{references.bib}
\DeclareBibliographyCategory{ignore}
\addtocategory{ignore}{recommendation} % adding recommendation to 'ignore' category so that it does not appear in the References
% Clickable references. Use \url{www.example.com} or \href{www.example.com}{description} to add a clicky url
\usepackage{nameref}
\urlstyle{same}

\DeclareFieldFormat{doi}{\url{https://doi.org/#1}}

% Include figures
\usepackage{graphbox}  % loads graphicx ppackage with extended options for vertical alignment of figures
% Improve typesetting in LaTex
\usepackage{microtype}
\DisableLigatures[f]{encoding = *, family = * }
% Text layout and font (Open Sans)
\setlength{\parindent}{0.4cm}
\linespread{1.2}
\RequirePackage[default,scale=0.90]{opensans}
% Defining document colors
\usepackage[table]{xcolor}
\definecolor{darkgray}{HTML}{808080}
\definecolor{mediumgray}{HTML}{6D6E70}
\definecolor{ligthgray}{HTML}{d9d9d9}
\definecolor{pciblue}{HTML}{74adca}
\definecolor{opengreen}{HTML}{77933c}
% Use adjustwidth environment to exceed text width
\usepackage{changepage}
% Adjust caption style
\usepackage[aboveskip=1pt,labelfont=bf,labelsep=period,singlelinecheck=off]{caption}


\usepackage[pdfborder={0 0 0},   
    colorlinks=true,
    linkcolor=blue,
    urlcolor=blue,
    citecolor=blue]{hyperref}  % sets link border to white




% DOI's
\newcommand{\DOIlink}{\href{https://doi.org/\DOI}{https://doi.org/\DOI}}
\newcommand{\DOIrecommendationlink}{\href{https://doi.org/\DOIrecommendation}{https://doi.org/\DOIrecommendation}}

\newcommand{\beginingpreprint}{
\vspace*{0.5cm}
\begin{flushleft}
\baselineskip=0pt

{\Huge
\fontseries{sb}\selectfont{\preprinttitle}}
\end{flushleft}
\vspace*{0.25cm}
\begin{flushleft}
\Large
\listauthors
\end{flushleft}
\bigskip
{\raggedright
\listinstitutions}
\\
\\
\textbf{Correspondence: } \href{mailto:\email}{\email}\\
\\
\vspace*{0.5cm}
\fcolorbox{pciblue}{pciblue}{
\parbox{\textwidth - 2\fboxsep}{
\vspace{0.25cm}
\begin{internallinenumbers}
\textbf{\large{\textsc{Abstract}}}\\

\preprintabstract\\

\textbf{\emph{Keywords: }}\preprintkeywords
\end{internallinenumbers}

\vspace{0.25cm}}
}
\newpage
\newgeometry{margin=1in}
}





\begin{document}
\beginingpreprint

%%%  [IF ACCEPTED] COMMENT linenumbers  %%%
\linenumbers
%%%%%%%%%%%%%%%%%%


%%%  TEXT OF THE PREPRINT  %%%
% use \emph{} for italics and \parencite{} or \textcite to cite a reference and \ref{XX} to cite the \label{XX}
% use \\ and a blank line to mark the end of paragraph and starting of a new one

\section*{\centering Introduction}

The discovery of the use of radiocarbon dating by \textcite{Libby1949} revolutionised the way archaeological remains were studied. (expand)

With the accumulation of dated archaeological remains, \textcite{Rick1987} introduces the ``Dates as Data'' approach. The idea behind this is to assume that changes in the abundance of anthropogenic samples the result from changes in the levels of human activity, which is also assumed to be proportional to human population size. Early applications used histograms of uncalibrated \textsuperscript{14}C ages while modern studies base their analysis in the 





review the methods and history of this approach and introduce/define SPD


There are, however, important limitations to this approach. For instance, \textcite{Williams2012} list five major challenges: (i) presence of biases associated to sampling within an archaeological site, (ii) limited number of samples available, (iii) influence of the calibration curve shape, that can create artificial concentration or overdispersion of dates, (iv) bias due to taphonomic loss and (v) the need to contrast results with independent data. 

Demographic inference from radiocarbon dates would benefit from a model-based statistical framework that could offer solutions to some of these problems. A probabilistic model of the data can allow incorporating processes that influence the data such as variation of atmospheric \textsuperscript{14}C through time (i.e. calibration) or the taphonomic loss. It would also allow to quantify uncertainty of estimates for small sample sizes and offer model-test tools that could be used to test specific hypothesis based on independent data.  

Unfortunately, the application of model-based inference from radiocarbon data is hindered by the fact that \textit{dates are not data}. Radiocarbon dating of an item is typically reported a single age value and an uncertainty quantification of that value (standard deviation). However, the true underlying data is the measure of the proportion of \textsuperscript{14}C/\textsuperscript{12}C in the sample, which is then transformed into a ``radiocarbon date'' following some machine-specific calibrations and assuming atmospheric \textsuperscript{14}C constant through time and geography. 




ABC offer a natural way of treating them as transformation of the raw data into summaries is an integral part of the ABC process.


This work proposes a model for the dated archaeological record. This formalization allows for a more objective interpretation of SPD curves and the application of simulation-based statistical inference.

 

\section*{\centering Material and methods}

\subsection*{A mathematical model for the dated archaeological record}

We describe the dated archaeological record (DAR) as a vector $\bm{R}$ that contains the number of items $R_t$ contributed by each year $t$ within a given range of dates $(t_{\mathrm{min}}, t_{\mathrm{max}})$. We consider that at each year $t$ there is a number of items $n_t$ that can potentially become part of the DAR. These are all the objects of organic origin associated to human activity. We assume a model in which each item in $n_t$ become part of $R_t$ with probability $p_t$. This probability $p_t$ describes a complex process, that includes the deposit of the item, its preservation through time, its discovery by archaeologists and the decision of archaeologists to use radiocarbon dating on it. This model represents a simplification of the reality: what constitutes a single item in $n_t$ can be ambiguous and the probability of an item to become part of the DAR will strongly depend on its nature (i.e. durability of the material, archaeological interest of the object). However, 






 The reduction of $n_t$ into $R_t$ is a relatively complex process that includes several ``steps'': 



 $\bm{R}$ depends on the number of objects that can potentially become part of the DAR ($\bm{n}$) and the probabilities for those objects to become part of the DAR. Assuming that this probability is the same for all items of a given year ($p_t$), we can use a Poisson distribution to model $\bm{R} \sim \mathrm{Pois}(\bm{\lambda})$, with $\lambda_t=n_tp_t$.
\\

In order to model the changes in the abundance of the DAR through time we can consider different functions to describe the relationship between $\lambda$ and $t$. The most simple case we can consider is that $\lambda$ is constant through time ($\lambda_t=\lambda_0$), but exponential ($\lambda_t=\lambda_0e^{rt}$) or logistic ($\lambda_t = \frac{K\lambda_0}{\lambda_0+(K-\lambda_0)e^{-rt}}$) change have also been previously used to fit changes of the DAR through time (REFERENCES). However, assuming that the same mathematical function governs the changes in $\lambda_t$ over large periods of time might not be appropriate (REF). Piecewise models, in which the whole range of time considered $(t_{\mathrm{min}},t_{\mathrm{max}})$ is divided in $m$ periods separated by $m+1$ times $t_0,t_1,\dots,t_{m}$ (with $t_0=t_{\mathrm{min}}$ and $t_{m}=t_{\mathrm{max}}$), can be used to set a different relationship between $\lambda$ and $t$ for each period. From this family of models, we will consider a piecewise exponential model defined by $m+1$ parameters $\lambda_{t_0},\lambda_{t_1},\dots ,\lambda_{t_{m}}$. Within each period $x\in(1,m)$, $\lambda$ changes exponentially with rate $r_x=\frac{\log\left(\lambda_{t_x}\right)-\log\left(\lambda_{t_{x-1}}\right)}{t_x-t_{x-1}}$.
\\

In some instances, the items of the DAR can be separated into different categories, such as geographical origin or nature of the material.
\\

\subsection*{Inference using approximate Bayesian computation}

If $\bm{R}$ were directly observable, it would be possible to compute the likelihood for different values of $\bm{\lambda}$ or the parameters that model the change of $\lambda_t$ through time, allowing for likelihood based estimates of these parameters. However, the true age of each item in the DAR is unknown. Radiocarbon dating provides the so-called uncalibrated dates. These uncalibrated dates would correspont to the true dates if the environment \textsuperscript{14}C/\textsuperscript{12}C proportions were constant through time and geography, and equal to that of the atmosphere in 1950, and if the true \textsuperscript{14}C halflife was Libby's estimate of 5568 \parencite{BronkRamsey2008}. In reality, data from the DAR has the form of vector $\bm{R'}$, which contains the number of items $R'_u$ dated at uncalibrated date $u$ within a range $(u_{\mathrm{min}},u_{\mathrm{max}})$.

The relationship between $\bm{R'}$ and $\bm{R}$ is given by calibration curves (REF), that can be used to transform an uncalibrated dates into a probability distributions for calibrated dates. These distributions are often multimodal and with a substantial uncertainty about the dates.

When aggregating all these probability distributions obtained from $\bm{R'}$ we can obtain the so-called sum of probability distributions (SPD). Still, the relationship between the SPD and $\bm{R}$ is unclear since individual $R_t$ values cannot be obtained from the SPD. In contrasts, given $\bm\lambda$ it is straight forward to simulate $\bm{R}$ and then $\bm{R'}$. 

\subsubsection*{Simulation of DAR}

The model of archaeological deposit rate can be used to generate samples of archaeological dates. However, these will not be comparable to radiocarbon dates, that represent measures of \textsuperscript{14}C rather than true dates. Based on the calibration curves these dates can be "uncalibrated" (sampling the radiocarbon date from a normal distribution with mean and variance taken from the appropriate year of the calibration data) to simulate the measurement of \textsuperscript{14}C, making it comparable with the observed data.




This process is similar to the simulation used by \textcite{Shennan2013}. The main difference is that the number of dates in our approach is not fixed to the observation but depends on $\theta_t$. This difference is important because the uncertainty is underestimated if the number of simulated dates is fixed.

\subsubsection*{Summary statistics}





\subsection*{Example of DAR: archaeological radiocarbon dates from Britain and Ireland}

In order to illustrate the methods developed in this work we reanalyse data of archaeological radiocarbon dates from Britain and Ireland \parencite{Bevan2017b}, first presented in \textcite{Bevan2017}. 




\begin{figure}
\center\includegraphics[width=10cm]{../results/Bevan_hist_all.pdf}
\center\includegraphics[width=10cm]{../results/Bevan_spd_all.pdf}
\caption{\textbf{Age distribution of DAR from \textcite{Bevan2017}.} some explanation here}
\label{fig:data}
\end{figure}


\section*{\centering Results}

\begin{table}[tbh]
\caption{\textbf{Model choice for full DAR from Britain and Ireland}}
\label{tab:model_choice}
\begin{tabular}{lrrrr}
\hline
model comparison & posterior probabilities & Bayes factor & chosen model & strength of evidence$^a$ \\
\hline\hline
constant \textit{vs.} exponential & (0.04, 0.96) & 23.55 & exponential & strong \\
exponential \textit{vs.} logistic & (0.09, 0.91) & 10.04 & logistic & strong \\
logistic \textit{vs.} piecewise & (0.03, 0.97) & 38.34 & piecewise & very strong \\
\hline
\end{tabular}
\footnotesize{$^a$ following \citeauthor{Jeffreys1961}' (\citeyear{Jeffreys1961}) scale.}\\
\end{table}


\section*{\centering Discussion}


Use ABC to combine with genetic data \parencite[supplementary figure 3 in ][]{Patterson2022}












Demographic inference from radiocarbon data assumes that there is a relationship between the magnitude of the population size ($N_t$) and the magnitude of dated archaeological samples at a given time $t$. Mathematically we can express this as $\theta_t \propto N_t$, with $\theta_t$ defined as rate at which items accumulate in the dated archaeological deposit. This parameter $\theta$ describes a complex reality: the reduction of the total amount of carbon-based items in the human settlement (wood, tools, food, humans themselves, etc.)\ to the amount that is deposited, then preserved though time, then excavated/discovered and finally dated. It has been acknowledged that each of these steps can be the source of biases: the amount of fire used (thus charcoal deposit) changes with climate, younger samples are more likely to be preserved than older ones, archaeological research questions drive which sites/periods are studied and which samples are dated \parencite{Rick1987,Williams2012}. However, it is assumed that these biases do not distort the general patterns and some attenuation procedures have also been proposed (CITAR BINING). Thus, from this point on the text will refer to the estimation of $\theta$ or its the rate of change ($r$) through time as ``demographic inference'' assuming no significant effects of these biases. Note, however, that $\theta$ and $r$ are not, strictly speaking, demographic parameters.

Modelling the dated archaeological deposit using $\theta$



Human activity leaves traces in various forms (tools, food, burials, etc.). I define the rate at which these items accumulate in the archaeological deposit as $\theta$. I consider that $\theta$ changes through time and $\theta_t$ is the rate for year $t$. Thus, the amount of items $n$ contributed to the archaeological deposit by year $t$ can be modeled as a Poisson distribution $n_t \sim P(\theta_t)$.

% TAPHONOMIC BIAS
% If we consider the possibility of taphonomic bias, only a proportion of those items will perdure into the present. The probability $p$ that any item remains in the archaeological deposit at present depends on time $p_t = e^{-\lambda t}$ (older items are more likely to disappear). The final number of items contributed to the archaeological record from year $t$ can be modeled with a Binomial distribution $n'_t \sim B(n_t,p_t)$

In this work it is assumed that $\theta_t \propto N_t$, that is, demography (population size, $N$) is the only factor determining the amount of items that constitute the archaeological record. From this point on the text will refer to the estimation of $\theta$ or its the rate of change through time as ``demographic inference''.




\section*{\centering Acknowledgements}

The idea of this work comes from a discussion with Mattias Jakobsson about the possibility of combining of the demographic inference used in archaeology with the approaches used in population genetics.

\section*{\centering Fundings}

This project has received funding from the European Union’s Horizon 2020 research and innovation programme under the Marie Skłodowska-Curie grant agreement No 791695 (TimeAdapt).

\section*{\centering Conflict of interest disclosure}

The authors of this article declare that they have no financial conflict of interest with the content of this article. Miguel de Navascués and Concetta Burgarella are recommenders for PCI Evolutionary Biology.

\section*{\centering Data, script and code availability}

Data are available online: DOI of the webpage hosting the data (eg \url{https://doi.org/10.24072/pcjournal.125} or other link)

\section*{\centering Supplementary information availability}

Script and codes are available online: DOI of the webpage hosting the script and codes (eg \url{https://doi.org/10.24072/pcjournal.125} or link)


\section*{\centering Supplementary Information}

\begin{table}[tbh]
\caption{\textbf{Notation}}
\label{tab:notation}
\begin{tabular}{ll}
\hline
symbol & meaning \\
\hline\hline
& \\
& \\
$n_t$ & number of object in year $t$ that can potentially become part of the DAR\\
& \\
& \\
$p_t$ & probability of an object to become part of the DAR at year $t$\\
& \\
$R_t$ & number of items in the DAR originating from year $t$\\
$R'_u$ & number of items in the DAR with radiocarbon date $u$\\
& \\
$t$ & time in years (calibrated)\\
$u$ & uncalibrated radiocarbon year\\
& \\
& \\
$\lambda_t$ & expected number of items in the DAR originating from year $t$\\
\hline
\end{tabular}
\end{table}

\titleformat*{\section}{\bfseries\Large\centering}

%%%%%% if they exist, DOIs are required %%%%%%%
\printbibliography[notcategory=ignore]

\end{document}
